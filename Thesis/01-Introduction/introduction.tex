\chapter{Introduction}

Several brain-machine interfaces have been proposed in recent years which aim to offer measurement of cortical information through minimally-invasive means. With the introduction of such data measurement techniques comes the need to further our understanding of the circuit-level characteristics of these neuronal systems such that the available data can be adequately utilised. The devices will allow access to information of neuronal networks at a level similar to that available from the analysis of man-made network systems, bridging the gap between biological neuronal analysis and classical communication theory that has been developed for, and applied to, these existing network systems \cite{mBarrosMolCom}.\\
The main aim of this project is to investigate the application of existing communication theory in the biological and neuronal domain, specifically in cortical neuronal circuits. Building a model based on this existing theory and machine learning techniques, we aim to infer internal network characteristics and classify cell types from limited endpoint data which could be measured through minimally-invasive approaches in the future, such as the Neural Dust proposed by researchers in the University of Berkeley \cite{NeurDust}. While these physical data measurement techniques are not yet mature enough to be used in practice, existing tools such as the NEURON simulation software from Yale can generate accurate simulated data based on known cell types, similar to the form of data which will be measurable by the aforementioned physical devices \cite{NEURON}.The data obtained from this software can then be processed through mathematical software such as Matlab or data-mining software such as RapidMiner, which offers a powerful environment for developing data processing and machine learning models \cite{rapidMiner}. By integrating and defining an interface between NEURON, Matlab, and RapidMiner, we obtain the framework needed to simulate the communication between biological neurons and to analyse it through machine learning techniques.
\par
One of the primary goals of this project is to investigate the classification and inference of neuronal circuits, especially at an inter-layer scale and with a larger variety of cell-types, progressing the field towards classification of more complex cortical circuits.\\
In this study, we investigate the effect of signal propagation in the circuits, and attempt to quantise the level of information communicated between cells. The data generated by the NEURON tool represents the signals as they exist in an ideal network, and so we investigate the effects that cell-to-cell connection parameters may have on the ability for the cells to transfer information between each other. To this end, we investigate the application of digital forms of information theoretic models in the biological domain, and find limitations in the insight that these models may provide in this setting.\\

We also investigate the use of classification algorithms in the analysis of various forms of cortical networks (either from known networks \cite{bbpTop} or using algorithmically driven approaches \cite{reimann2015algorithm}. One requirement in the use of such classifiers is the need to reduce the dimensionality of the measured data. This is done through feature extraction, which can also be applied to the neural network and decision tree models to improve performance. We therefore attempt to characterise the response of individual neural cells in a finite dimension space. Through the use of characteristic models and integration with high-performance classifiers, we develop a classification system to predict cell type (and sub-group type) based entirely off input-output voltage measurements.\\

\par

Finally, we investigate the use of the classification models trained on the data from the previous step in the application of network tomography to reconstruct the cell-types in a known topology based on endpoint measurements taken around the network.\\
The project aims can therefore be summarised as the following objectives:
\begin{itemize}
    \item Construct and simulate a number of cortical circuits in NEURON, extracting data for each circuit.
    \item Using existing models in the domain of information theory, analyse the entropy and mutual information of cell-to-cell connections.
    \item Using Matlab and RapidMiner, construct models to investigate the application of cell-characterisation models to extract features from the endpoint measurements for the training of a number of classification algorithms.
    \item Investigate the application of the trained classification model to predict unknown cell-types from the endpoint measurements of a cortical network.
\end{itemize}


\section{Outline of Proceeding Chapters}

The proceeding document is laid out as follows. In Chapter \ref{chap:back}, \emph{Background}, we cover the supporting theory of the topics and domains relating to the experimentation and investigations conducted in the course of this study, such as the biological theory of neurons and neuronal circuits, the theory and application of network tomography, the concept of information and information theory, and finally a comparison of the functionality of various classification algorithms. In Chapter \ref{chap:relWork}, \emph{Related Work}, we review studies related to the work carried out in this investigation, comparing the methods in which the work was conducted and the results found. In Chapter \ref{chap:meth}, \emph{Methodology} we describe the steps taken to carry out the experiments conducted in the various experiments, including a complete description of the supporting tools and frameworks used, as well as the methods in which the generated data was processed into usable information. In Chapter \ref{chap:res}, \emph{Results}, we present and describe the results generated from the experimentation and analysis. In Chapter \ref{chap:disc}, \emph{Discussion and Future Work}, we discuss the generated results, comparing our results with those of the related work and discussing the reasoning behind any differences between the two, while also suggesting potential future work that could improve on our investigation. Finally, we conclude in Chapter \ref{chap:conc}, \emph{Conclusion}, with a summary of the contributions and findings of the investigation.

